\documentclass{beamer}
\mode<presentation> {
\usepackage{color}
\definecolor{bottomcolour}{rgb}{0.21,0.11,0.21}
\definecolor{middlecolour}{rgb}{0.21,0.11,0.21}
\setbeamercolor{structure}{fg=white}
\setbeamertemplate{frametitle}[default]%[center]
\setbeamercolor{normal text}{bg=black, fg=white}
\setbeamertemplate{background canvas}[vertical shading]
[bottom=bottomcolour, middle=middlecolour, top=black]
\setbeamertemplate{items}[circle]
\setbeamertemplate{navigation symbols}{} %no nav symbols
\setbeamercolor{block title}{use=structure,fg=white,bg=structure.fg!50!red!50!blue!100!green}
\setbeamercolor{block body}{parent=normal text,use=block title,bg=block title.bg!5!white!10!bg,fg=white}
\setbeamertemplate{navigation symbols}{}
}
\usepackage{graphicx} 
\usepackage{booktabs} 
\usepackage[utf8]{inputenc}  
\usepackage[T1]{fontenc}  
\usepackage{geometry}     
%\usepackage[francais]{babel} 
\usepackage{eurosym}
\usepackage{verbatim}
\usepackage{ragged2e}
\justifying

\input{cc_beamer}

\title[Journalisme et Chiffrement]{Journalisme et Chiffrement
\\~\\
\includegraphics[scale=0.4]{./images/CELSA.jpg}
} 
\date{Paris - 27 février 2016}
\author{Genma}

\begin{document}

%% Titlepage
\begin{frame}
	\titlepage
	\vfill
	\begin{center}
		\CcGroupByNcSa{0.83}{0.95ex}\\[2.5ex]
		{\tiny\CcNote{\CcLongnameByNcSa}}
		\vspace*{-2.5ex}
	\end{center}
\end{frame}

\begin{frame}
\frametitle{\includegraphics[scale=0.4]{./images/Genma.jpg} \ \ \  A propos de moi  }
\begin{columns}[c] 
\column{.55\textwidth} 
\textbf{Où me trouver sur Internet?}
\begin{itemize}
\item Le Blog de Genma : http://genma.free.fr
\item Twitter : http://twitter.com/genma
\end{itemize}
\column{.5\textwidth} 
\includegraphics[scale=0.40] {./images/blog.png} 
\end{columns}
\end{frame}


%------------------------------------------------
\begin{frame}
\frametitle{Programme}

\justifying{
Pourquoi ne dit-on pas crypter ? Un peu de théorie
\begin{itemize}
\item Le principe du chiffrement
\item Le chiffrement symétrique (Cesar)
\item Le chiffrement asymétrique (les enveloppes) (Mails et GPG)
\end{itemize}
}

\justifying{
Cacher des documents sur son ordi, le coffre-fort numérique
\begin{itemize}
\item Le coffre-fort numérique avec TrueCrypt/Veracrypt
\end{itemize}
}

Pause
\justifying{
Naviguer sur le web sans être vu, Tor
\begin{itemize}
\item Comment l’installer, comment ça marche…
\end{itemize}
}

\end{frame}
%-----------------------------------------------

%-----------------------------------------------
\begin{frame}
\begin{center}
\Huge{Pourquoi ne dit-on pas crypter ? Un peu de théorie}
\end{center}
\end{frame}


%------------------------------------------------
\begin{frame}
\frametitle{Le principe du chiffrement}


\begin{block}{Le chiffrement}
\justifying{
Le chiffrement consiste à chiffrer un document/un fichier à l’aide d’une clef de chiffrement. 
L’opération inverse étant le déchiffrement. 
}
\end{block}
\begin{block}{Le cryptage}
\justifying{
Le terme \emph{cryptage} est un anglicisme, tiré de l’anglais encryption. 
Le décryptage existe : il s’agit de "casser" un document chiffré lorsqu’on n’en a pas la clef.
}
\end{block}
\begin{block}{La cryptographie}
\justifying{
La science quant-à elle s’appelle  la "cryptographie".
}
\end{block}
\end{frame}

%------------------------------------------------
\begin{frame}
\frametitle{Le chiffrement symétrique (Cesar)}


\begin{block}{Le chiffrement symétrique}
\justifying{
Cela consiste à chiffrer un message avec la même clef que celle qui sera utilisé pour le déchiffrement. 
Exemple : le code de César avec un décalage de lettres. A-C, B-D etc.
\\
Nous venons en paix -  Pqwu xgpqpu gp rckz
\\
On applique le processus inverse pour avoir le message.
}
\end{block}

\begin{block}{Une clef de chiffrement c'est quoi?}
\justifying{

Une clef s'appelle une clef car elle ouvre/ferme le cadenas qu'est l'algorithme de chiffrement utilisé.
\begin{itemize}
\item Ici, l'algorithme est dans la notion de décalage.
\item La clef est le nombre de lettre décallées (ici deux lettres).
\end{itemize}
}
\end{block}

\end{frame}
%------------------------------------------------
\begin{frame}
\frametitle{Le chiffrement asymétrique 1/2}

\begin{block}{Clef publique - clef privée}
\justifying{

Le chiffrement asymétrique repose sur le couple clef publique - clef privée.
\\$\Rightarrow$  Ce qu'il faut comprendre/retenir : 
\begin{itemize}
\item Ma clef privée est secrète.
\item Ma clef publique est distribuée à tous.
\end{itemize}
}
\end{block}

\begin{block}{L'algorithme de chiffrement}
\justifying{
L'algorithme de chiffrement est bien plus complexe que le fait de décaler des lettres ; il repose sur des notions mathématiques (nombre premiers...)
}
\end{block}

\end{frame}

\begin{frame}
\frametitle{Le chiffrement asymétrique 2/2}

\begin{block}{Le chiffrement}
Avec la clef publique de mon correspondant, je chiffre  un fichier.
\\$\Rightarrow$ Le fichier ne peut plus être déchiffré que par la personne qui possède la clef privée correspondant à la clef publique que j'ai utilisée (donc mon correspondant).
\end{block}

\begin{block}{Le déchiffrement}
Avec sa clef privée,  mon correspondant déchiffre le fichier.
\\
$\Rightarrow$ Il peut alors lire le message.
\end{block}

\begin{block}{Cas concret}
Le chiffrement de ses mails avec PGP.
\end{block}
\end{frame}

%-----------------------------------------------

%-----------------------------------------------
\begin{frame}
\begin{center}
\Huge{Le chiffrement en pratique}
\end{center}
\end{frame}

%------------------------------------------------
\begin{frame}
\frametitle{ Le coffre-fort numérique avec TrueCrypt/Veracrypt}

\begin{center}
\includegraphics[scale=0.4] {./images/Truecrypt18.png}
\end{center}
\end{frame}

%----------------------------------------------------------------------------------------
\begin{frame}
\frametitle{Les enveloppes}

\begin{center}
\includegraphics[scale=0.4] {./images/gpg.jpg}
\end{center}
\end{frame}

%----------------------------------------------------------------------------------------
\begin{frame}
\frametitle{Les boîtes 1/2}

\begin{columns}[c] 
\column{.55\textwidth} 
\includegraphics[scale=0.28] {./images/GPGBox_02.png}
\column{.5\textwidth} 
\includegraphics[scale=0.28] {./images/GPGBox_03.png}
\end{columns}
\end{frame}

%----------------------------------------------------------------------------------------
\begin{frame}
\frametitle{Les boîtes 2/2}

\includegraphics[scale=0.65] {./images/GPGBox_01.jpg}
\end{frame}


%-----------------------------------------------
\begin{frame}
\begin{center}
\Huge{Aller plus loin? Tor et le TorBrowser}
\end{center}
\end{frame}

%----------------------------------------------------------------------------------------
\begin{frame}
\begin{center}
\Huge{Quelques mots sur Tor ? }
\\~\\ \includegraphics[scale=0.4]{./images/logo_tor.jpg}
\end{center}
\huge{Attention : la présentation \emph{complète} dure une bonne heure et demie...}
\end{frame}

%----------------------------------------------------------------------------------------
\begin{frame}
\frametitle{Comment fonctionne Tor ?}
\begin{center}
\includegraphics[keepaspectratio,width=\textwidth, height=.8\textheight]{images/tor-keys1}
\end{center}
\end{frame}

%----------------------------------------------------------------------------------------
\begin{frame}
\frametitle{Tor et les enveloppes}

\begin{center}
\includegraphics[scale=0.3] {./images/tor.jpg}
\end{center}
\end{frame}

%----------------------------------------------------------------------------------------
\begin{frame}
\frametitle{A quoi sert TOR?}

\begin{block}{Ce que l'usage de Tor permet de faire}
\justifying{
\begin{itemize}
\justifying{
\item  d'échapper au fichage publicitaire,
\item  de publier des informations sous un pseudonyme,
\item  d'accéder à des informations en laissant moins de traces,
\item  de déjouer des dispositifs de filtrage (sur le réseau de son entreprise, de son Université, en Chine ou en France…),
\item  de communiquer en déjouant des dispositifs de surveillance,
\item  de tester son pare-feu,
\item  … et sûrement encore d'autres choses.
}
\end{itemize}
$\Rightarrow$ Tor dispose également d'un système de "services cachés" qui permet de fournir un service en cachant l'emplacement du serveur.
}
\end{block}
\end{frame}

%----------------------------------------------------------------------------------------
\begin{frame}
\frametitle{Télécharger le Tor Browser}
\justifying{
Toutes les versions (dans différentes langues, différents OS) sont disponibles sur le site du projet : 
\\ \url{https://www.torproject.org/}
\\ Rq : Il existe la possibilité de le recevoir par mail...
}
\begin{center}
\includegraphics[scale=0.5]{./images/tor2.jpg}
\end{center}
\end{frame}

%----------------------------------------------------------------------------------------
\begin{frame}
\frametitle{Lancer le Tor Browser}
\begin{center}
\includegraphics[scale=0.3]{./images/tor_browser03.jpg}
\end{center}
\end{frame}
%----------------------------------------------------------------------------------------

\begin{frame}
\frametitle{Utiliser Tor - Tails}
\justifying{
Tails (The Amnesic Incognito Live System) est un système d'exploitation complet basé sur Linux et Debian, en live.
}
\begin{center}
\includegraphics[scale=0.3]{./images/tails.jpg}
\\~\\
\url{https://tails.boom.org}
\end{center}
\end{frame}

%----------------------------------------------------------------------------------------
\begin{frame}
\begin{center}
\includegraphics[scale=0.4] {./images/LogoCafeViePrivee.jpg}
\end{center}
\end{frame}

%----------------------------------------------------------------------------------------
\begin{frame}
\Huge{\centerline{Merci de votre attention.}}
\Huge{\centerline{Place aux questions.}}
\end{frame}

%----------------------------------------------------------------------------------------
\begin{frame}
\frametitle{\includegraphics[scale=0.4]{./images/Genma.jpg} \ \ \  Me contacter?}
\Huge{\centerline{Le Blog de Genma}}
\Huge{\centerline{http://genma.free.fr}}
\Huge{\centerline{~}}
\Huge{\centerline{Twitter : @genma}}
\end{frame}

%============================================================================================
\begin{frame}
\Huge{\centerline{ANNEXES}}
\end{frame}

\end{document}